\documentclass{homework}

\title{Homework 1}

% x 4.1.9
% x 4.1.10
% 3 4.1.14 (a)
% 4 4.5.6
% x 4.5.7
% x 4.5.12
% x 4.5.15

\begin{document}

\maketitle

\begin{problem}{1}
%%%% %%%% %%%% %%%% %%%% %%%% %%%% %%%% %%%% %%%% %%%% %%%% %%%% %%%% %%%% %%%%|
Note that $\mathcal{D}_N$ can be divided into four cases:
\begin{itemize}
\item $0 \leq x, y < 1$.
      $A_N = \{(i, j) \in \mathcal{D}_N | 0 \leq i/2^N, j/2^N < 1\}$.
\item The top line of the square.
      $T_N = \{(i, 2^N) \in \mathcal{D}_N\}$.

\item The right line of the square.
      $R_N = \{(2^N, j) \in \mathcal{D}_N\}$.

\item All the rest, on which $f$ is always $0$.
\end{itemize}
Then, the upper sum and lower sum can be written as:
\begin{align*}
U_N(f) &= \sum_{C \in A_N} M_C(f) \mathrm{vol}_n C \\
       &+ \sum_{C \in T_N} M_C(f) \mathrm{vol}_n C \\
       &+ \sum_{C \in R_N} M_C(f) \mathrm{vol}_n C \\
       &= \sum_{(i, j) \in A_N}
          \sin(\frac{i - j}{2^N} + \frac{1}{2^N}) \frac{1}{2^{2N}}
        + 0 \\
       &+ \sum_{(i, j) \in R_N}
          \sin(1 - \frac{j}{2^N}) \frac{1}{2^{2N}}
\end{align*}
\begin{align*}
L_N(f) &= \sum_{C \in A_N} m_C(f) \mathrm{vol}_n C \\
       &= \sum_{(i, j) \in A_N}
          \sin(\frac{i - j}{2^N} - \frac{1}{2^N}) \frac{1}{2^{2N}} \\
\end{align*}
%%%% %%%% %%%% %%%% %%%% %%%% %%%% %%%% %%%% %%%% %%%% %%%% %%%% %%%% %%%% %%%%|
\begin{align*}
U_N(f) - L_N(f)
&= \sum_{(i, j)\in A_N}\sin(\frac{i-j}{2^N}+\frac{1}{2^N})\frac{1}{2^{2N}}
 - \sum_{(i, j)\in A_N}\sin(\frac{i-j}{2^N}-\frac{1}{2^N})\frac{1}{2^{2N}} \\
&+ \sum_{(i, j) \in R_N} \sin(1 - \frac{j}{2^N}) \frac{1}{2^{2N}} \\
&= \frac{1}{2^{2N}} \sum_{(i, j)\in A_N}
   2\cos(\frac{i-j}{2^N})\sin(\frac{1}{2^N})
 + \sum_{(i, j) \in R_N} \sin(1 - \frac{j}{2^N}) \frac{1}{2^{2N}} \\
&\leq \frac{1}{2^{2N}} \sum_{(i, j)\in A_N}
      2\cos(\frac{i-j}{2^N})\sin(\frac{1}{2^N})
    + \frac{1}{2^N} \cdot 1 \cdot \frac{1}{2^{2N}}
\end{align*}
Also, note that
$$\frac{1}{2^{2N}} \sum_{(i, j)\in A_N}
  2\cos(\frac{i-j}{2^N})\sin(\frac{1}{2^N})
  \leq U_N(f) - L_N(f)
$$
. Since $\sin(\frac{1}{2^N})$ goes to $0$ as $N$ goes to infinity, we have
$$\lim_{N \to \infty}
  \frac{1}{2^{2N}} \sum_{(i, j)\in A_N}
  2\cos(\frac{i-j}{2^N})\sin(\frac{1}{2^N})
  = 0
$$
. Moreover,
$$\lim_{N \to \infty} \frac{1}{2^N} = 0$$
. Therefore, by squeeze theorem,
$$\lim_{N \to \infty} U_N(f) - L_N(f) = 0$$
\QED
\end{problem}

\begin{problem}{2}
\begin{enumerate}
\item
%%%% %%%% %%%% %%%% %%%% %%%% %%%% %%%% %%%% %%%% %%%% %%%% %%%% %%%% %%%% %%%%|
\begin{align*}
4U_1(f) &= [(\frac{1}{2})^2 + (\frac{1}{2})^2] + [(\frac{1}{2})^2 + 1^2]
         + [1^2 + (\frac{1}{2})^2] + (1^2 + 1^2) \\
        &= \frac{1}{2} + \frac{5}{4} + \frac{5}{4} + 2 \\
        &= 5 \\
 U_1(f) &= \frac{5}{4}
\end{align*}
\begin{align*}
4L_1(f) &= [(\frac{1}{2})^2 + (\frac{1}{2})^2] + [(\frac{1}{2})^2 + 0^2]
         + [0^2 + (\frac{1}{2})^2] + (0^2 + 0^2) \\
        &= \frac{1}{2} + \frac{1}{4} + \frac{1}{4} + 0 \\
        &= 1 \\
 L_1(f) &= \frac{1}{4}
\end{align*}

\item
%%%% %%%% %%%% %%%% %%%% %%%% %%%% %%%% %%%% %%%% %%%% %%%% %%%% %%%% %%%% %%%%|
\begin{align*}
\int_T f\pmat{x\\y} \mathrm{d}x \mathrm{d}y
&= \int_0^1(\int_0^1 x^2+y^2 \mathrm{d}y) \mathrm{d}x \\
&= \int_0^1 (x^2y + \frac{y^3}{3}) |_0^1 \mathrm{d}x \\
&= \int_0^1 x^2 + \frac{1}{3} \mathrm{d}x \\
&= (\frac{x^3}{3} + \frac{1}{3}x)|_0^1 \\
&= \frac{2}{3}
\end{align*}
, which is between the upper and the lower sum.
\end{enumerate}
\end{problem}

\begin{problem}{5}
%%%% %%%% %%%% %%%% %%%% %%%% %%%% %%%% %%%% %%%% %%%% %%%% %%%% %%%% %%%% %%%%|
Let $R = \{(x,y,z) \in \mathbb{R}^3 | x,y,z\geq0, x+2y+3z \leq 1\}$,
\begin{align*}
\int_R xyz\ \mathrm{d}x\mathrm{d}y\mathrm{d}z
&=\int_0^1(\int_0^{1-x}(\int_0^{1-x-2y}xyz\ \mathrm{d}z)\mathrm{d}y)\mathrm{d}x
\\
&=\int_0^1(\int_0^{1-2y}(\int_0^{1-2y-x}xyz\ \mathrm{d}z)\mathrm{d}x)\mathrm{d}y
\\
&=\int_0^1(\int_0^{1-3z}(\int_0^{1-3z-x}xyz\ \mathrm{d}y)\mathrm{d}x)\mathrm{d}z
\end{align*}
\end{problem}

\begin{problem}{6}
%%%% %%%% %%%% %%%% %%%% %%%% %%%% %%%% %%%% %%%% %%%% %%%% %%%% %%%% %%%% %%%%|
Let $R = \{(x, y) \in \mathbb{R}^2 | 0 \leq x \leq a, x^2 \leq y \leq a^2\}$,
\begin{align*}
\int_0^a (\int_{x^2}^{a^2} \sqrt{y}e^{-y^2} \mathrm{d}y) \mathrm{d}x
&= \int_R \sqrt{y}e^{-y^2} \mathrm{d}x \mathrm{d}y \\
&= \int_0^{a^2} (\int_{0}^{\sqrt{y}} \sqrt{y}e^{-y^2} \mathrm{d}x) \mathrm{d}y \\
&= \int_0^{a^2} ye^{-y^2} \mathrm{d}y \\
&= -\frac{1}{2}e^{-y^2} |_0^{a^2} \\
&= -\frac{1}{2}e^{-a^4} + \frac{1}{2}
\end{align*}
\end{problem}

\begin{problem}{7}
%%%% %%%% %%%% %%%% %%%% %%%% %%%% %%%% %%%% %%%% %%%% %%%% %%%% %%%% %%%% %%%%|
The region can be seen as the union of two-dimensional balls located at
$(0,0,z)$, with radius $\sqrt{z}$ when $0 \leq z \leq 5$ and $\sqrt{10-z}$
when $5 \leq z \leq 10$. Its volume can be integrated with the circle area
function:
\begin{align*}
V &= 2\int_0^5 \pi z \mathrm{d}z \\
  &= \int_0^5 2\pi z \mathrm{d}z \\
  &= \pi z^2 |^5_0 \\
  &= 25\pi
\end{align*}
\end{problem}

\end{document}
