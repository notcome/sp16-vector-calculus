\documentclass{homework}

\title{Homework 6}

\DeclareMathOperator{\sgn}{sgn}
\allowdisplaybreaks

\begin{document}
\maketitle

% 6.4.1
\begin{problem}{1}
  $$\vec{\nabla} f\pmat{x \\ y \\ z} = \bmat{2x \\ 2y \\ -2z}$$
  $$[\mathbf{D} \gamma] = \bmat{
    \cos\theta & -r\sin\theta \\
    \sin\theta &  r\cos\theta \\
    1          &  0
  }$$
  \begin{align*}
    \det [\vec{\nabla} f \cdot \gamma, \mathbf{D}_1 \gamma, \mathbf{D}_2 \gamma]
    &= \det \bmat{
      2r\cos\theta & \cos\theta & -r\sin\theta \\
      2r\sin\theta & \sin\theta &  r\cos\theta \\
      -2r          & 1          &  0
    } \\
    &= -4r^2 < 0
  \end{align*}
  Hence $\gamma$ does not preserve orientation.
\end{problem}

% 6.4.4
\begin{problem}{2}
  Let $U = [0, 1] \times [0, 1] \times [0, 1]$.
  Let $\gamma: U \to S$ be
  $$\gamma\pmat{x_1 \\ x_2 \\ x_3} = \pmat{x_1 \\ x_2 \\ x_3 \\ x_1x_2x_3}$$
  \begin{align*} &
    \int_S x_3\ dx_1 \land dx_2 \land dx_4 |d^3x| \\ =&
    \int_U
      \sgn dx_1 \land dx_2 \land dx_3 \cdot
      x_3\ dx_1 \land dx_2 \land dx_4 \left(
      P_{\pmat{u_1 \\ u_2 \\ u_3 \\ u_1u_2u_3}}
        \bmat{
          1      & 0      & 0 \\
          0      & 1      & 0 \\
          0      & 0      & 1 \\
          u_2u_3 & u_1u_3 & u_1u_2
        }\right)
      |d^3u| \\ =&
    \int_U u_3\det\bmat{
      1      & 0      & 0 \\
      0      & 1      & 0 \\
      u_2u_3 & u_1u_3 & u_1u_2}
      |d^3u| \\ =&
    \int_U u_3(u_1u_2 - u_1u_3) |d^3u| \\ =&
    \int_0^1 \left(
      \int_0^1 \left(
        \int_0^1 u_3(u_2-u_3)u_1\ du_1
        \right) du_2
      \right) du_3 \\ =&
    \dfrac{1}{2}
    \int_0^1 \left( \int_0^1 u_3(u_2 - u_3)\ du_2 \right) du_3 \\ =&
    \dfrac{1}{2} \left(
      \int_0^1\int_0^1 u_3u_2\ du_2\ du_3 -
      \int_0^1 u_3^2\ du_3
    \right) \\ =&
    \dfrac{1}{2}(\dfrac{1}{4} - \dfrac{1}{3}) \\ =&
    -\dfrac{1}{24}
  \end{align*}
\end{problem}

% 6.5.4
\begin{problem}{3}
  Note
  $$\bvec{F} \times \bvec{G} = \bmat{
    F_2G_3 - F_3G_2 \\
    F_3G_1 - F_1G_3 \\
    F_1G_2 - G_2F_1
  }$$
  Therefore,
  \begin{align*}
    \Phi_{\bvec{F} \times \bvec{G}} &=
    F_2\ dy\land G_3\ dz - F_3\ dz\land G_2\ dy \\ &-
    F_3\ dx\land G_1\ dz + F_1\ dz\land G_3\ dx \\ &+
    F_1\ dx\land G_2\ dy - G_2\ dx\land F_1\ dy \\ &+
    0\ dx \land dx + 0\ dy \land dy + 0\ dz \land dz  \\ &=
    \sum_{i=1}^3\sum_{j=1}^3 F_i\ d_i\land G_j\ d_j \\ &=
    W_{\bvec{F}} \land W_{\bvec{G}}
  \end{align*}
  \QED
\end{problem}

% 6.5.5
\begin{problem}{4}
  \begin{align*}
    W_{\bvec{F}} \land \Phi_{\bvec{G}} &=
    (F_1\ dx + F_2\ dy + F_3\ dz) \land
    (G_1\ dy \land dz + G_2\ dz \land dx + G_3\ dx \land dy) \\ &=
    (F_1G_1 + F_2G_2 + F_3G_3) dx \land dy \land dz \\ &=
    \bvec{F} \cdot \bvec{G}\ dx \land dy \land dz \\ &=
    M_{\bvec{F} \cdot \bvec{G}}
  \end{align*}
  Since dot product is commutative,
  $ M_{\bvec{F} \cdot \bvec{G}}
  = M_{\bvec{G} \cdot \bvec{F}}
  = W_{\bvec{G}} \land \Phi_{\bvec{F}}$ \QED
\end{problem}

% 6.5.6
\begin{problem}{5}
  \begin{align*}
    W_{\bvec{F}}(P_{\bpnt{a}}(\bvec{u})) &=
    F(\bpnt{a}) \cdot \bvec{u} \\ &=
    \bmat{0 \\ -1 \\ -2} \cdot \bmat{1 \\ -1 \\ 1} \\ &=
    1 - 2 = -1
  \end{align*}
\end{problem}

% 6.5.9
\begin{problem}{6}
  \begin{enumerate}
    \item
      One spanned by $\bmat{1 \\ 1 \\ 0}, \bmat{0 \\ 0 \\ 1}$, which evaluates
      to $\det\bmat{1 & 0 \\ 0 & 1} + \det\bmat{1 & 0 \\ 0 & 1} = 2$.
    \item
      Positive: $\pmat{0 \\ 0 \\ -1}$. This evaluates to
      $-(-1) \det\bmat{1 & 0 \\ 0 & 1} = 1$.

      Negative: $\pmat{0 \\ 0 \\ 1}$. This evaluates to
      $-1 \det\bmat{1 & 0 \\ 0 & 1} = -1$.
  \end{enumerate}
\end{problem}

% 6.5.13
\begin{problem}{7}
  \textbf{Multilinearity}.
  Since the determinant is linear as a function of each column $\bvec{v}_i$,
  $\Phi_{\bvec{F}(\bpnt{x})}$ also is.

  \textbf{Antisymmetry}.
  Switching any two columns of $\Phi_{\bvec{F}(\bpnt{x})}$ will switch
  the corresponding $\bvec{v}_i$ and $\bvec{v}_j$ of the determinant,
  which will change the sign due to the antisymmetry of determinant. \QED
\end{problem}

% 6.5.15
\begin{problem}{8}
  $$\bvec{F}(\bpnt{x}) = \bmat{1 \\ 0 \\ -1}$$
  $$\bvec{f}(\bpnt{x}) = -2$$
  \begin{enumerate}
    \item
    $$W_{\bvec{F}}(P_{\bpnt{x}}(\bvec{v}_1))
    = \bmat{1 \\ 0 \\ -1} \cdot \bmat{0 \\ 1 \\ 1}
    = -1$$

    \item
    $$\Phi_{\bvec{F}}(P_{\bpnt{x}}(\bvec{v}_1, \bvec{v}_2))
    = 0$$

    \item
    $$M_f(P_{\bpnt{x}}(\bvec{v}_1, \bvec{v}_2, \bvec{v}_3))
    = -2 \det \bmat{0 & 1 & -1 \\ 1 & 1 & 1 \\ 1 & 0 & 1} = -2$$
  \end{enumerate}
\end{problem}

% 6.5.17
\begin{problem}{9}
  Four parametrizations can be used for each side of the rectangle:
  \begin{align*}
    \gamma_1(t) &= \pmat{0 \\ t},\quad 0 \leq t \leq a \\
    \gamma_2(t) &= \pmat{t \\ a},\quad 0 \leq t \leq a \\
    \gamma_3(t) &= \pmat{b \\ t},\quad 0 \leq t \leq a \\
    \gamma_4(t) &= \pmat{t \\ 0},\quad 0 \leq t \leq a \\
  \end{align*}
  The later two parametrizations reverse the orientation.
  Integrate them:
  \begin{align*}
    \int_0^a \bmat{0 \\ t} \cdot \bmat{0 \\ 1} dt     &= \dfrac{a^2}{2} \\
    \int_0^b \bmat{ta \\ ae^t} \cdot \bmat{1 \\ 0} dt &= \dfrac{ab^2}{2} \\
    \int_a^0 \bmat{bt \\ te^b} \cdot \bmat{0 \\ 1} dt &= -\dfrac{a^2e^b}{2} \\
    \int_b^0 \bmat{0 \\ 0} \cdot \bmat{1 \\ 0} dt     &= 0 \\
  \end{align*}
  The sum is
  $$\dfrac{a^2}{2} + \dfrac{ab^2}{2} - \dfrac{a^2e^b}{2}$$
\end{problem}

% 6.5.18
\begin{problem}{10}
  \begin{align*}
    \int_{\gamma(U)} W_{\bvec{F}} &=
    \int_0^\alpha
      (x^2\ dx + y^2\ dy + z^2\ dz) \left(
        P_{\pmat{\cos t \\ \sin t \\ at}}
          \bmat{-\sin t \\ \cos t \\ a}
      \right) dt \\ &=
    \int_0^\alpha (- \cos^2 t\sin t
                   + \sin^2 t\cos t
                   + a^3 t^2) dt \\ &=
    \dfrac{1}{3} (a^3\alpha^3 + \sin^3 a + \cos^3 a - 1)
  \end{align*}
\end{problem}

% 6.5.20
\begin{problem}{11}
  Let $\gamma: U \to \mathbb{R}^3$ be
  $$\gamma \pmat{\theta \\ r} = \pmat{r\cos\theta \\ r\sin\theta \\ r}$$
  , where $U = [0, 2\pi] \times [0, 1]$.
  
  Pick
  $$\bvec{n} \pmat{\theta \\ r} = \pmat{r\cos\theta \\ r\sin\theta \\ -r}$$
  which is an outward normal vector field, it is easy to check that $\gamma$ is
  orientation-preserving: the determinant evaluates to $(1+r)r^2$, which is
  positive for all $r \in [0, 1]$.
  \begin{align*}
    \int_{\gamma{U}} \Phi_{\bvec{F}} &=
    \int_U (x\ dy \land dz - y\ dx \land dz + z\ dx \land dy) \left(
      P_{\pmat{r\cos\theta \\ -r\sin\theta \\ r^2\cos\theta\sin\theta}}
        \left(
          \bmat{-r\sin\theta \\ r\cos\theta \\ 0},
          \bmat{  \cos\theta \\ \sin\theta  \\ 1}
        \right)
      \right) |d\theta\ dr| \\ &=
    \int_0^{2\pi} \left(
      \int_0^1 (\cos^2\theta - \sin^2\theta - \cos\theta\sin\theta)r^2\ dr
    \right) d\theta \\ &=
    \dfrac{1}{3} \int_0^{2\pi}
      (\cos^2\theta - \sin^2\theta - \cos\theta\sin\theta) d\theta \\ &=
    0
  \end{align*}
\end{problem}

\end{document}
