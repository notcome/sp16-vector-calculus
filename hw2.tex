\documentclass{homework}

\title{Homework 2}

\begin{document}

\maketitle

\begin{problem}{1}
%%%% %%%% %%%% %%%% %%%% %%%% %%%% %%%% %%%% %%%% %%%% %%%% %%%% %%%% %%%% %%%%|
\begin{align*}
\det\bmat{1&-2&3&0\\4&0&1&2\\5&-1&2&1\\3&2&1&0}
&= \det\bmat{0&1&2\\&-1&2&1\\&2&1&0} +
  2\det\bmat{4&1&2\\5&2&1\\3&1&0} +
  3\det\bmat{4&0&2\\5&-1&1\\3&2&0} \\
&= (-\det\bmat{-1&1\\2&0} + 2\det\bmat{-1&2\\2&1}) \\
&+ (8\det\bmat{2&1\\1&0} - 2\det\bmat{5&1\\3&0} + 4\det\bmat{5&2\\3&1}) \\
&+ (12\det\bmat{-1&1\\2&0} + 6\det\bmat{5&-1\\3&2}) \\
&= (2-10)+(-8+6-4)+(-24+78) \\
&= -8-6+54 \\
&= 40
\end{align*}
\end{problem}

\begin{problem}{2}
%%%% %%%% %%%% %%%% %%%% %%%% %%%% %%%% %%%% %%%% %%%% %%%% %%%% %%%% %%%% %%%%|
Let
\begin{align*}
A &= \bmat{a_1&a_2\\a_3&a_4} \\
B &= \bmat{b_1&b_2\\b_3&b_4}
\end{align*}
. We have
\begin{align*}
AB &= \bmat{a_1b_1 + a_2b_3 & \hdots \\ \hdots & a_3b_2 + a_4b_4} \\
BA &= \bmat{b_1a_1 + b_2a_3 & \hdots \\ \hdots & b_3a_2 + b_4a_4}
\end{align*}
. Clearly,
\begin{align*}
\mathrm{tr}(AB) &=
a_1b_1 + a_2b_3 + a_3b_2 + a_4b_4 \\
&=
b_1a_1 + b_2a_3 + b_3a_2 + b_4a_4 \\
&=
\mathrm{tr}(BA)
\end{align*}
\end{problem}

\begin{problem}{3}
\begin{enumerate}
\item 
%%%% %%%% %%%% %%%% %%%% %%%% %%%% %%%% %%%% %%%% %%%% %%%% %%%% %%%% %%%% %%%%|
\begin{align*}
\det[\hdots, \bvec{0}, \hdots]
&= \det[\hdots, \bvec{v}, \hdots] - \det[\hdots, \bvec{v}, \hdots] \\
&= 0
\end{align*}
, where $\bvec{v}$ denotes an arbitrary compatible vector. \QED

\item 
%%%% %%%% %%%% %%%% %%%% %%%% %%%% %%%% %%%% %%%% %%%% %%%% %%%% %%%% %%%% %%%%|
Assume $A = [\hdots, \bvec{v}_i, \hdots, \bvec{v}_j, \hdots]$
where $\bvec{v}_i = \bvec{v}_j$. Then,
\begin{align*}
\det[\hdots, \bvec{v}_i, \hdots, \bvec{v}_j, \hdots]
&=- [\hdots, \bvec{v}_j, \hdots, \bvec{v}_i, \hdots] \\
\det[\hdots, \bvec{v}_i, \hdots, \bvec{v}_j, \hdots]
  + [\hdots, \bvec{v}_j, \hdots, \bvec{v}_i, \hdots]
&= 0 \\
2\det[\hdots, \bvec{v}_i, \hdots, \bvec{v}_j, \hdots]&= 0 \\
\det[\hdots, \bvec{v}_i, \hdots, \bvec{v}_j, \hdots]&= 0
\end{align*}
\end{enumerate}
. This completes the proof. \QED
\end{problem}

\begin{problem}{4}
%%%% %%%% %%%% %%%% %%%% %%%% %%%% %%%% %%%% %%%% %%%% %%%% %%%% %%%% %%%% %%%%|
First, one needs to show that matrix multiplication ``respects'' the three
properties. For clarity, I use $M_i$ to denote a matrix $M$'s $i$th row and
$M^i$ its $i$th column. Fix $A$.
\begin{enumerate}
\item Multilinearity.
%%%% %%%% %%%% %%%% %%%% %%%% %%%% %%%% %%%% %%%% %%%% %%%% %%%% %%%% %%%% %%%%|
$$AB = \bmat{A_1\cdot B^1 & \hdots & A_1\cdot B^i & \hdots & A_1\cdot B^n \\
             \vdots       & \vdots & \vdots       & \vdots & \vdots \\
             A_n\cdot B^1 & \hdots & A_n\cdot B^i & \hdots & A_n\cdot B^n}$$
If $B^i = \alpha\bvec{u} + \beta\bvec{w}$, then since
$$AB_{ji} = A_j\cdot(\alpha\bvec{u} + \beta\bvec{w})
= A_j\cdot\alpha\bvec{u} + A_j\cdot\beta\bvec{w}$$
, we have
\begin{align*}
f(B) &= \dfrac{1}{\det A} \det(AB) \\
&= \dfrac{1}{\det A}\det
\bmat{A_1\cdot B^1 & \hdots & A_1\cdot\alpha\bvec{u} & \hdots & A_1\cdot B^n \\
      \vdots       & \vdots & \vdots       & \vdots & \vdots \\
      A_n\cdot B^1 & \hdots & A_n\cdot\alpha\bvec{u} & \hdots & A_n\cdot B^n}\\
&+ \dfrac{1}{\det A}\det
\bmat{A_1\cdot B^1 & \hdots & A_1\cdot\beta\bvec{w} & \hdots & A_1\cdot B^n \\
      \vdots       & \vdots & \vdots       & \vdots & \vdots \\
      A_n\cdot B^1 & \hdots & A_n\cdot\beta\bvec{w} & \hdots & A_n\cdot B^n}\\
&= f([B^1,\hdots,\alpha\bvec{u},\hdots,B^n])
 + f([B^1,\hdots,\beta \bvec{w},\hdots,B^n])
\end{align*}

\item Antisymmetry.
%%%% %%%% %%%% %%%% %%%% %%%% %%%% %%%% %%%% %%%% %%%% %%%% %%%% %%%% %%%% %%%%|
\begin{align*}
& f([\hdots,\bvec{u},\hdots,\bvec{w},\hdots]) + 
  f([\hdots,\bvec{w},\hdots,\bvec{u},\hdots])
\\ =&
\dfrac{1}{\det A}\det[\hdots,A\bvec{u},\hdots,A\bvec{w},\hdots] \\ +&
\dfrac{1}{\det A}\det[\hdots,A\bvec{w},\hdots,A\bvec{u},\hdots]
\\ =& 0
\end{align*}

\item Normalization.
%%%% %%%% %%%% %%%% %%%% %%%% %%%% %%%% %%%% %%%% %%%% %%%% %%%% %%%% %%%% %%%%|
\begin{align*}
f(I) &= \dfrac{\det(AI)}{\det A} \\
     &= \dfrac{\det A}{\det A} \\
     &= 1
\end{align*}
\end{enumerate}
Therefore, $f$ has to be the determinant function, and
$\det B = f(B) = \dfrac{\det(AB)}{\det A}$. \QED
\end{problem}

\begin{problem}{5}
%%%% %%%% %%%% %%%% %%%% %%%% %%%% %%%% %%%% %%%% %%%% %%%% %%%% %%%% %%%% %%%%|
\newcommand \comPerm[2]{\langle#1,#2\rangle}
\newcommand \sgn{\mathrm{sgn}}
For $\sigma_1 \in \mathrm{Perm}_n$ and $\sigma_2 \in \mathrm{Perm}_m$,
denote $\comPerm{\sigma_1}{\sigma_2} \in \mathrm{Perm_{n+m}}$ as a point-wise
composition of two permutations, where $\sigma_1$ works on the first $n$ entries
and $\sigma_2$ works on the last $m$ entries. Formally speaking,
$$\comPerm{\sigma_1}{\sigma_2} = \mathrm{shift}_n(\sigma_2) \circ \sigma_1$$
where
$$\mathrm{shift}_n(\sigma)(x) = \sigma(x - n) + n$$
. Since $\mathrm{shift}_n$ preserves signature,
$$\sgn(\comPerm{\sigma_1}{\sigma_2}) = \sgn(\sigma_1)\sgn(\sigma_2)$$
From the determinant formula
$$\det A = \sum_{\sigma \in \mathrm{Perm}_{n}}
\sgn(\sigma)a_{1,\sigma(1)}\hdots a_{n,\sigma(n)}$$
and the fact that the matrix in question is of form
$$\bmat{A&|&C\\0&|&B}$$
, we should note that if a permutation contributes to the sum, its first
$n$ values must be less or equal than $n$---otherwise there will be a $0$
that cancels everything---and its last $m$ values must range from $n + 1$
to $n + m$. In other words, for every $\sigma \in \mathrm{Perm}_{n+m}$,
$\sigma = \comPerm{\sigma_1}{\sigma_2}$. The formula can therefore be
rewritten as
\begin{align*}
\det \bmat{A&|&C\\0&|&B} &=
\sum_{\comPerm{\sigma_1}{\sigma_2} \in \mathrm{Perm}_{n+m}}
\sgn(\sigma_1)\sgn(\sigma_2)
A_{1,\sigma_1(1)}\hdots A_{n,\sigma_1(n)}
B_{1,\sigma_1(1)}\hdots B_{m,\sigma_2(m)} \\
&=
\sum_{\sigma_1 \in \mathrm{Perm}_n}
\sgn(\sigma_1)A_{1,\sigma_1(1)}\hdots A_{n,\sigma_1(n)}(
\sum_{\sigma_2 \in \mathrm{Perm}_m}
\sgn(\sigma_2)B_{1,\sigma_2(2)}\hdots B_{m,\sigma_2(m)}) \\
&= \det B \sum_{\sigma_1 \in \mathrm{Perm}_n}
\sgn(\sigma_1)A_{1,\sigma_1(1)}\hdots A_{n,\sigma_1(n)} \\
&= \det A \det B
\end{align*}
. This completes the proof. \QED
\end{problem}

\begin{problem}{6}
%%%% %%%% %%%% %%%% %%%% %%%% %%%% %%%% %%%% %%%% %%%% %%%% %%%% %%%% %%%% %%%%|
\newcommand \sgn{\mathrm{sgn}}
\begin{enumerate}
\item $\sigma_1$ is the identity function. Since
$\sgn(\sigma_1 \circ \sigma) = \sgn(\sigma_1)\sgn(\sigma_1)$, its signature
has to be $1$.

\item $\sigma_4, \sigma_5, \sigma_6$ are transpositions, so their signatures
are $-1$.

\item $\sigma_2 = \sigma_6 \circ \sigma_5$, so its signature is $1$.

\item $\sigma_3 = \sigma_6 \circ \sigma_4$, so its signature is $1$.
\end{enumerate}
\end{problem}

\begin{problem}{7}
\begin{enumerate}
\item
%%%% %%%% %%%% %%%% %%%% %%%% %%%% %%%% %%%% %%%% %%%% %%%% %%%% %%%% %%%% %%%%|
Use $\lambda_1, \hdots, \lambda_n$ to denote $A$'s diagonal entries.
$$\det(tI - A) = (t-\lambda_1) \hdots (t-\lambda_n)$$
$$\chi_A(X) = (X-\lambda_1 I) \hdots (X-\lambda_n I)$$
Substitute $A$:
\begin{align*}
\chi_A(A) &= (A-\lambda_1 I) \hdots
\bmat{\lambda_1\\&\ddots\\&&0\\&&&\lambda_n}
\bmat{\lambda_1\\&\ddots\\&&\lambda_{n-1}\\&&&0} \\
&= (A-\lambda_1 I) \hdots
\bmat{\lambda_1\\&\ddots\\&&0\\&&&0} \\
&= [0]
\end{align*}

\item
%%%% %%%% %%%% %%%% %%%% %%%% %%%% %%%% %%%% %%%% %%%% %%%% %%%% %%%% %%%% %%%%|
\begin{align*}
\chi_{P^{-1}BP} &=
\det(tI - P^{-1}BP) \\ &=
\det(P^{-1}tIP - P^{-1}BP) \\ &=
\det(P^{-1}(tI - B)P) \\ &=
\det(tI - B) \\
&= \chi_B
\end{align*}

Let $A$ be a diagonalizable matrix and $D = P^{-1}AP$ be a diagonal matrix.
\begin{align*}
\chi_A(A) = \chi_A(PDP^{-1}) &=
(PDP^{-1})^n + c_1(PDP^{-1})^{n-1} + \hdots + c_n \\ &=
P(D^n + c_1D^{n-1} + \hdots + c_n)P^{-1} \\ &=
P\chi_B(D)P^{-1} \\ &=
P\chi_D(D)P^{-1} \\ &= [0]
\end{align*}

\item
%%%% %%%% %%%% %%%% %%%% %%%% %%%% %%%% %%%% %%%% %%%% %%%% %%%% %%%% %%%% %%%%|
By theorem 4.8.24, there exists a sequence of diagonalizable matrices that
converges to some arbitrary square matrix $A$. Also, $det$ is continuous. Let
$$f(X) = \chi_X(X)$$
. Denote the sequence as $\{A_i\}$, then $\{f(A_i)\}$ also converges. It
follows that
$$f(A) = \lim_{i \to \infty} f(A_i) = [0]$$
. This completes the proof. \QED
\end{enumerate}
\end{problem}
\end{document}
