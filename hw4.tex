\documentclass{homework}

\title{Homework 4}

\DeclareMathOperator{\img}{img}
\DeclareMathOperator{\rank}{rank}
\DeclareMathOperator{\vol}{vol}

\begin{document}

\maketitle

% 5.3.1
\begin{problem}{1}
\begin{enumerate}
% 5.3.1.a
\item
Let $\gamma$ be the corresponding parametrization in the Cartesian coordinates:
$$\gamma(t) = \bmat{r(t)\cos(\theta(t))\\r(t)\sin(\theta(t))}$$
$$[\mathbf{D}\gamma(t)] = \bmat{
r'(t)\cos(\theta(t))-r(t)\sin(\theta(t))\theta'(t)\\
r'(t)\sin(\theta(t))+r(t)\cos(\theta(t))\theta'(t)
}$$
The length of the piece of the curve $l$ is
\begin{align*}
l  &=
\int_a^b \sqrt{\det([\mathbf{D}\gamma(t)]^{\top}[\mathbf{D}\gamma(t)])}dt
\\ &=
\int_a^b \sqrt{
(r'(t)\cos(\theta(t))-r(t)\sin(\theta(t))\theta'(t))^2+
(r'(t)\sin(\theta(t))+r(t)\cos(\theta(t))\theta'(t))^2}dt
\\ &=
\int_a^b \sqrt{(r'(t))^2+(r(t)\theta'(t))^2}dt
\end{align*}

% 5.3.1.b
\item
Substitute $r(t)$ and $\theta(t)$, we obtain
\begin{align*}
l  &=
\int_0^b \sqrt{\alpha^2e^{-2\alpha t}+e^{-2\alpha t}} dt
\\ &=
\sqrt{\alpha^2+1}\dfrac{1-e^{-\alpha b}-1}{\alpha}
\end{align*}
\begin{align*}
\lim_{\alpha \to 0}
\sqrt{\alpha^2+1}\dfrac{1-e^{-\alpha b}}{\alpha}
&=
\lim_{\alpha \to 0}
\sqrt{\alpha^2+1}\dfrac{1-(1-\alpha b+\alpha^2b^2/2 - \hdots)}{\alpha} \\
&=
\lim_{\alpha \to 0}
\sqrt{\alpha^2+1}(b+\alpha(-\alpha b^2/2 + \hdots)) \\
&= b
\end{align*}

% 5.3.1.c
\item
Since $\theta(t) = t$ and every $2\pi$ change in $\theta(t)$ makes the spiral
turn around the origin once, the spiral turns infinitely many times around the
origin as $t \to \infty$.
$$\lim_{t \to \infty}l = \dfrac{\sqrt{\alpha^2+1}}{\alpha}(1-e^{-\alpha t})
= \dfrac{\sqrt{\alpha^2+1}}{\alpha}$$
\end{enumerate}
\end{problem}

% 5.3.3
\begin{problem}{2}
\begin{enumerate}
\item
$$\gamma(t)=\pmat{
r(t)\cos\theta(t)\cos\varphi(t)\\
r(t)\sin\theta(t)\cos\varphi(t)\\
r(t)\sin\varphi(t)}$$
\begin{align*}
l  &=
\int_a^b \sqrt{\det([\mathbf{D}\gamma(t)]^{\top}[\mathbf{D}\gamma(t)])}dt
\\ &=
\int_a^b \sqrt{(r'(t))^2+(r(t)\varphi'(t))^2+(r(t)\cos\varphi(t)\theta'(t))^2}dt
\end{align*}

\item
\begin{align*}
l  &=
\int_0^a \sqrt{(-\sin{t})^2+(\cos{t})^2+(\cos^2t\frac{1}{\cos^2t})^2}dt
\\ &= \int_0^a \sqrt{2} dt
\\ &= \sqrt{2}a
\end{align*}
\end{enumerate}
\end{problem}

% 5.3.5
\begin{problem}{3}
\begin{enumerate}
\item
$$\gamma(r\\\theta)=\pmat{2r\cos\theta\\3r\sin\theta\\r^2}$$
\begin{align*}
S  &=
\int_0^{2\pi}\int_0^a \sqrt{\det([D\gamma]^{\top}[D\gamma])}drd\theta
\\ &=
\int_0^{2\pi}\int_0^a \sqrt{4r^2\sin^2\theta+9r^2\cos^2\theta+9}drd\theta
\end{align*}

\item
Consider a function of $z$ that yields each horizontal slice of the region,
given by the area formula for ellipse that is calculated in the early homework:
$$f(z) = 2\sqrt{z}\cdot3\sqrt{z}\cdot\pi=6\pi z$$
$$\vol{R}=\int_0^{a^2}f(z)dz=3\pi z|_0^{a^2}=3\pi a^4$$
\end{enumerate}
\end{problem}

% 5.3.6
\begin{problem}{4}
Use the cylindrical coordinates for parametrization:
$$\gamma\pmat{r\\\theta} = \pmat{r\cos\theta\\r\sin\theta\\r^2}$$
$$[\mathbf{D}\gamma\bmat{r\\\theta}]=
\bmat{\cos\theta&-r\sin\theta\\
      \sin\theta& r\cos\theta\\
      2r        & 0}$$
$$[\mathbf{D}\gamma\bmat{r\\\theta}]^{\top}[\mathbf{D}\gamma\bmat{r\\\theta}]
=\bmat{1+4r^2&0\\0&r^2}$$
The integral can be rewritten as
\begin{align*}
\int_S(x^2+y^2+3z^2) |d^2\bvec{x}|
&=
\int_0^3 \int_0^{2\pi}(r^2+3r^4)\sqrt{\det{\bmat{1+4r^2&0\\0&r^2}}}drd\theta
\\ &=
\int_0^3 \int_0^{2\pi}(r^2+3r^4)\sqrt{r^2+4r^4}drd\theta
\\ &=
\int_0^3 2\pi(r^2+3r^4)\sqrt{r^2+4r^4}dr
\end{align*}
With some ugly substitutions, the result is
$$(\dfrac{1}{105}+\dfrac{138787}{210}\sqrt{37})\pi$$
\end{problem}

% 5.3.8
\begin{problem}{5}
Using the same parameterization as in the last problem, we have
\begin{align*}
S  &=
\int_0^1\int_0^{2\pi}\sqrt{r^2+4r^4}d\theta dr
\\ &=
\int_0^1 2\pi\sqrt{r^2+4r^4}dr
\\ &= \dfrac{1}{6}(5\sqrt{5}-1)\pi
\end{align*}
\end{problem}

% 5.3.13.a
\begin{problem}{6}
Let $\gamma$ be the horizontal radial projection:
$$\gamma\bmat{\theta\\z}=\bmat{
\sqrt{1-z^2}\cos\theta\\
\sqrt{1-z^2}\sin\theta\\
z}$$
Then,
\begin{align*}
\vol{S_2} &= \int_{S_1} \sqrt{\det([\mathbf{D}\gamma]^{\top}[\mathbf{D}\gamma])}
|d^2\bvec{x}| \\ &=
\int_{S_1} 1|d^2\bvec{x}| \\ &= \vol{S_1}
\end{align*}
\end{problem}

% 5.3.15
\begin{problem}{7}
\begin{enumerate}
\item
First, check if $\img(\gamma) \subset S^3$. Note the first three entry of the
parametrization is the parametrization for $S^2$ in $\mathbb{R}^3$ multiplying
$\cos\varphi$. Hence, the square sum of the four entries is
$$\cos^2\psi + \sin^2\psi = 1$$

Next, check if the domain of $\gamma$. As noted above, $\theta$ and $\varphi$
``constitutes" the parametrization for $S^2$ with the appropriate domains.
Therefore, if one fixes $\sin\psi$, for any $x_1,x_2,x_3$ such that
$(x_1,x_2,x_3,\sin\psi)\in S^3$, there exists one and only one---except for
certain sets of measure zero, which can be safely ignored---pair of
$\theta$ and $\varphi$ such that
$\gamma(\theta,\varphi,\psi)=(x_1,x_2,x_3,\sin\psi)$. Moreover,
$f(\psi)=\sin\psi$ from $[-\pi/2,\pi/2]$ to $[-1,1]$ is indeed a bijective
function.

This completes the proof that $\gamma$ is a parametrization for $S^3$. \QED
\item
$$\det([\mathbf{D}\gamma]^{\top}[\mathbf{D}\gamma]) = \cos^4\psi\cos^2\varphi$$
\begin{align*}
\vol{S_3} &=
\int_{-\pi/2}^{\pi/2}\int_{-\pi/2}^{\pi/2}\int_{0}^{2\pi}
\sqrt{\cos^4\psi\cos^2\varphi}d\theta d\varphi d\psi \\ &= 2\pi^2
\end{align*}
\end{enumerate}
\end{problem}

% 6.1.3
\begin{problem}{8}
\begin{enumerate}
\item $\det\bmat{1&1\\2&2}=0$
\item $\det\bmat{1&-2\\0&1}+2\det\bmat{0&1\\1&0}=-1$
\item $\det\bmat{2&2\\0&-3}=-6$
\item $\det\bmat{1&-2&-2\\3&1&2\\3&1&2}=0$
\end{enumerate}
\end{problem}

% 6.1.6
\begin{problem}{9}
Only a and f makes sense.
\begin{enumerate}
\item
\begin{align*}
dx_1 \land dx_2
\pmat{\bmat{1\\0\\1},\bmat{2\\3\\1}}
&= \det\pmat{1&0\\2&3} \\
&= 3
\end{align*}

\item
\begin{align*}
dx_1 \land dx_2 \land dx_3
\pmat{\bmat{1\\0\\3},\bmat{3\\7\\2},\bmat{2\\0\\1}}
&= \det\pmat{1&3&2\\0&7&0\\3&2&1} \\
&= -35
\end{align*}
\end{enumerate}
\end{problem}

% 6.1.8
\begin{problem}{10}
\begin{enumerate}
\item $0$

\item $2e^2$

\item $4\det\bmat{3&-1&-1\\2&1&-1\\1&0&1} = 28$
\end{enumerate}
\end{problem}

% 6.1.10
\begin{problem}{11}
$$\varphi = a_1dx_2\land dx_3 - a_2dx_1\land dx_3 + a_3dx_1\land d_x2$$
\end{problem}

% 6.1.11
\begin{problem}{12}
\begin{align*}
\varphi\land\psi(\bvec{v}_1,\bvec{v}_2,\bvec{v}_3,\bvec{v}_4) &=
\varphi(\bvec{v}_1,\bvec{v}_2)\psi(\bvec{v}_3,\bvec{v}_4)-
\varphi(\bvec{v}_1,\bvec{v}_3)\psi(\bvec{v}_2,\bvec{v}_4)+
\varphi(\bvec{v}_1,\bvec{v}_4)\psi(\bvec{v}_2,\bvec{v}_3)\\&+
\varphi(\bvec{v}_2,\bvec{v}_3)\psi(\bvec{v}_1,\bvec{v}_4)-
\varphi(\bvec{v}_2,\bvec{v}_4)\psi(\bvec{v}_1,\bvec{v}_3)+
\varphi(\bvec{v}_3,\bvec{v}_4)\psi(\bvec{v}_1,\bvec{v}_2)
\end{align*}
\end{problem}

% 6.1.12
\begin{problem}{13}
\begin{enumerate}
\item Let $\varphi, \psi$ be two 1-forms. Then
\begin{align*}
(\varphi \land \psi)(\bvec{v}_1,\bvec{v}_2) &=
\varphi(\bvec{v}_1)\psi(\bvec{v}_2)-
\varphi(\bvec{v}_2)\psi(\bvec{v}_1) \\ &=
-(\psi(\bvec{v}_1\varphi(\bvec{v}_2))
-\psi(\bvec{v}_2)\varphi(\bvec{v}_1)) \\ &=
-(\psi \land \varphi)(\bvec{v}_1,\bvec{v}_2)
\end{align*}

\item Let $\varphi$ and $\psi$ be 1-form and 2-form respectively. Then
\begin{align*}
(\varphi \land \psi)(\bvec{v}_1,\bvec{v}_2,\bvec{v}_3) &=
\varphi(\bvec{v}_1)\psi(\bvec{v}_2,\bvec{v}_3)-
\varphi(\bvec{v}_2)\psi(\bvec{v}_1,\bvec{v}_3)+
\varphi(\bvec{v}_3)\psi(\bvec{v}_1,\bvec{v}_2)\\ &=
\psi(\bvec{v}_1,\bvec{v}_2)\varphi(\bvec{v}_3)-
\psi(\bvec{v}_1,\bvec{v}_3)\varphi(\bvec{v}_2)+
\psi(\bvec{v}_2,\bvec{v}_3)\varphi(\bvec{v}_1)\\ &=
(\psi \land \varphi)(\bvec{v}_1,\bvec{v}_2,\bvec{v}_3)
\end{align*}
\end{enumerate}
\end{problem}

\end{document}