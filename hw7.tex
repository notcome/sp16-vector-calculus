\documentclass{homework}

\title{Homework 7}

\DeclareMathOperator{\sgn}{sgn}
\DeclareMathOperator{\grad}{grad}
\DeclareMathOperator{\curl}{curl}
\DeclareMathOperator{\rediv}{div}
\renewcommand\div{\rediv}
\allowdisplaybreaks

\begin{document}
\maketitle

% 6.6.5 (skip part b)
\begin{problem}{1}
  \begin{enumerate}
    \item
      For any $\bpnt{x} \in P$,
      $$\bvec{v} = \bmat{1 \\ 0 \\ -1}, \bvec{w} = \bmat{0\\ 1 \\ -1}$$
      constitute a basis for $T_{\bpnt{x}}P$.
      \begin{align*}
        &\sgn \det[\vec{N}, \bvec{v}, \bvec{w}] = \sgn 3 = 1 \\
        &\sgn dx \land dy (\bvec{v}, \bvec{w}) = 1 \\
        &\sgn dx \land dz (\bvec{v}, \bvec{w}) = -1 \\
        &\sgn dy \land dz (\bvec{v}, \bvec{w}) = 1 \\
      \end{align*}
      Therefore, $\sgn dx \land dy$ and $\sgn dy \land dz$ gives the same
      orientation of $P$ as $\vec{N}$.
    \item
      It is sufficient to check whether $\gamma$ is orientation preserving.
      Since
      $$\det[\vec{N}, \bvec{v}_{\mathrm{out}(t)}, \gamma'(t)] =
        \det[\vec{N}, \gamma(t), \gamma'(t)] = -\sqrt{3}$$
      $\gamma$ is orientation reversing, which makes it incompatible with the
      boundary orientation of $\partial X$.
    \item
      First, check whether the signs of the three forms acting on $\gamma'$ are
      continuous:
      \begin{align*}
        &dx(P_{\gamma(t)} \gamma'(t))
          = -\dfrac{\sin t}{\sqrt{2}} -\dfrac{\cos t}{\sqrt{6}} \\
        &dy(P_{\gamma(t)} \gamma'(t))
          =  \dfrac{\sin t}{\sqrt{2}} -\dfrac{\cos t}{\sqrt{6}} \\
        &dz(P_{\gamma(t)} \gamma'(t))
          = 2\dfrac{\cos t}{\sqrt{6}}
      \end{align*}
      None of them does not change signs within $0 \leq t \leq 2\pi$, hence they
      cannot define the orientation at every point.
    \item
    \begin{align*}
      &x\ dy - y\ dx (P_{\gamma(t)} \gamma'(t)) = -1/\sqrt{3} \\
      &x\ dz - z\ dx (P_{\gamma(t)} \gamma'(t)) =  1/\sqrt{3} \\
      &y\ dz - z\ dy (P_{\gamma(t)} \gamma'(t)) = -1/\sqrt{3}
    \end{align*}
    Since the parametrization is orientation reversing, only forms that
    give a negative the result, namely the first and the third one,
    define the boundary orientation.
  \end{enumerate}
\end{problem}

% 6.7.3
\begin{problem}{2}
  \begin{enumerate}
    \item
      Let $\vec{F}_2 = \pmat{F_1 \\ F_2}$.
      \begin{align*}
        \mathbf{d}\Phi_{\vec{F}_2} &=
        \mathbf{d}(-F_2\ dx + F_1 dy) \\ &=
        \mathbf{d}\left( \dfrac{-y}{x^2+y^2} dx \right) +
        \mathbf{d}\left( \dfrac{x}{x^2+y^2} dx \right) \\ &=
        \dfrac
          {(y^2-x^2)dy \land dx + (x^2 + y^2 - 2x^2)dx \land dy}
          {(x^2+y^2)^2} \\ &= 0
      \end{align*}
    \item
      Let $\vec{F}_3 = \pmat{F_1 \\ F_2 \\ F_3}$.
      \begin{align*}
        \mathbf{d}\Phi_{\vec{F}_3} &=
        \mathbf{d}(F_1\ dy \land dz - F_2\ dx \land dz + F_3 dx \land dy) \\ &=
        \dfrac
          {(x^2+y^2+z^2)^{3/2} - 3x^2(x^2+y^2+z^2)^{3/2}}
          {(x^2+y^2+z^2)^3} dx \land dy \land dz \\ &-
        \dfrac
          {(x^2+y^2+z^2)^{3/2} - 3y^2(x^2+y^2+z^2)^{3/2}}
          {(x^2+y^2+z^2)^3} dy \land dx \land dz \\ &+
        \dfrac
          {(x^2+y^2+z^2)^{3/2} - 3z^2(x^2+y^2+z^2)^{3/2}}
          {(x^2+y^2+z^2)^3} dz \land dx \land dy \\ &=
        \dfrac
          {3(x^2+y^2+z^2)^{3/2} - 3(x^2+y^2+z^2)^{3/2}}
          {(x^2+y^2+z^2)^3} \\ &= 0
      \end{align*}
  \end{enumerate}
\end{problem}

% 6.7.4
\begin{problem}{3}
  $$\mathbf{d}\varphi
  = 2x_1x_3\ dx_1 \land dx_2 \land dx_3
  - x_1\ dx_1 \land dx_3 \land dx_4
  $$
\end{problem}

% 6.7.6
\begin{problem}{4}
  \begin{enumerate}
    \item
      $\mathbf{d}(f\ dx \land dz) = -(D_2f)\ dx \land dy \land dz$
    \item
      $\mathbf{d}(f\ dy \land dz) = (D_1f)\ dx \land dy \land dz$
  \end{enumerate}
\end{problem}

% 6.7.9
\begin{problem}{5}
  To satisfy the equation, we have
  \begin{flalign*}
    &&D_3q(x, z) = -x &&\\
    &&D_3p(y, z) = -y &&\\
    &&D_1q(x, z) = D_2p(y, z) && \text{to cancel } dx \land dy
  \end{flalign*}
  Therefore, $\omega$ is of the form
  $$\omega = -(yz + k_1)\ dx - (xz + k_2)\ dy$$
\end{problem}

% 6.7.11
\begin{problem}{6}
  \newcommand \ED{\mathbf{d}}
  \begin{enumerate}
    \item
      It follows immediately from the product rule.
    \item
      Assume that we have a $k$-form $\varphi_0$ and a $l$-form $\psi_0$.
      $\varphi_0$ can be written as the sum of $\varphi_1 + \varphi_2$ where
      both forms are of the form of $\varphi$ stated in the problem. The same
      holds true for $\psi_0$.

      Next, it is sufficient to show that the theorem works for
      $\varphi_0 \land \psi_0$ if it works for
      $\varphi_1, \varphi_2, \psi_1, \psi_2$:
      \begin{align*}
        \ED((\varphi_1 + \varphi_2) \land (\psi_1 + \psi_2)) &=
        \ED(\varphi_1 \land \psi_1) + \ED(\varphi_1 \land \psi_2) +
        \ED(\varphi_2 \land \psi_1) + \ED(\varphi_2 \land \psi_2) \\ &=
        \ED\varphi_1 \land \psi_1 + (-1)^k\varphi_1 \land \ED\psi_1 \\ &+
        \ED\varphi_1 \land \psi_1 + (-1)^k\varphi_1 \land \ED\psi_2 \\ &+
        \ED\varphi_1 \land \psi_2 + (-1)^k\varphi_1 \land \ED\psi_1 \\ &+
        \ED\varphi_1 \land \psi_2 + (-1)^k\varphi_1 \land \ED\psi_2 \\ &=
        (\ED \varphi_1 + \ED \varphi_2) \land (\psi_1 + \psi_2) +
        (-1)^k (\varphi_1 \land \ED(\psi_1 + \psi_2) +
                \varphi_2 \land \ED(\psi_1 + \psi_2)) \\ &=
        \ED(\varphi_1 + \varphi_2) \land (\psi_1 + \psi_2) +
        (-1)^k (\varphi_1 + \varphi_2) \land \ED(\psi_1 + \psi_2)
      \end{align*}
    \item
      Let $a = a(\bpnt(x)), b = b(\bpnt{x})$.
      \begin{align*}
        \ED(\varphi \land \psi) &=
        \ED(ab) \land dx_{i_1} \land \hdots \land dx_{i_k}
                \land dx_{j_1} \land \hdots \land dx_{j_l} \\ &=
        a \ED b
          \land dx_{i_1} \land \hdots \land dx_{i_k}
          \land dx_{j_1} \land \hdots \land dx_{j_l} \\ &+
        b \ED a
          \land dx_{i_1} \land \hdots \land dx_{i_k}
          \land dx_{j_1} \land \hdots \land dx_{j_l} \\ &=
        a dx_{i_1} \land \hdots \land dx_{i_k} \land
        (-1)^k\ED b dx_{j_1} \land \hdots \land dx_{j_l} \\ &+
        \ED \varphi \land
        b dx_{j_1} \land \hdots \land dx_{j_l} \\ &=
        \ED \varphi \land \psi +
        (-1)^k \varphi \land \ED \psi
      \end{align*}
  \end{enumerate}
\end{problem}

% 6.8.6
\begin{problem}{7}
  \begin{enumerate}
    \item
      $$\vec{F}\pmat{x \\ y \\ z} = \pmat{x \\ -xy \\ xy}$$
    \item
      \newcommand \ED{\mathbf{d}}
      \begin{align*}
        \ED \Phi_{\vec{F}} &= dx \land dy \land dz + y\ dy \land dx \land dz \\
                           &= (1-x) dx \land dy \land dz
      \end{align*}
      \begin{align*}
        M_{\div\vec{F}} &=
        \vec{\nabla} \cdot \vec{F}\ dx \land dy \land dz \\ &=
        (D_1F_1 + D_2F_2 + D_3F_3) dx \land dy \land dz \\ &=
        (1 - x + 0) dx \land dy \land dz \\ &= \ED \Phi_{\vec{F}}
      \end{align*}
  \end{enumerate}
\end{problem}

% 6.8.11 (part a only)
\begin{problem}{8}
  \begin{align*}
    \div \bmat{x^2-y \\ -2yz \\ x^3y^2} = 2xy - 2z
  \end{align*}
  \begin{align*}
    \div \bmat{\sin{xz} \\ \cos{yz} \\ xyz} = z\cos{xz}-z\sin{yz}+xy
  \end{align*}
  \begin{align*}
    \curl \bmat{x^2-y \\ -2yz \\ x^3y^2} &=
    \bmat{D_1 \\ D_2 \\ D_3} \times \bmat{x^2-y \\ -2yz \\ x^3y^2} \\ &=
    \bmat{2x^3+y+2y \\ -3x^2y^2 \\ -x^2}
  \end{align*}
  \begin{align*}
    \curl \bmat{\sin{xz} \\ \cos{yz} \\ xyz} &=
    \bmat{D_1 \\ D_2 \\ D_3} \times \bmat{\sin{xz} \\ \cos{yz} \\ xyz} \\ &=
    \bmat{xz + y\sin{yz} \\ x\cos{xz} - yz \\0}
  \end{align*}

\end{problem}

% 6.8.12
\begin{problem}{9}
  \begin{align*}
    \grad \div \bvec{F} - \curl \curl \bvec{F} &=
    \bmat{
      D_1D_1F_1 + D_1D_2F_2 + D_1D_3F_3 \\
      D_2D_1F_1 + D_2D_2F_2 + D_2D_3F_3 \\
      D_3D_1F_1 + D_3D_2F_2 + D_3D_3F_3
    } \\ &-
    \bmat{
      D_2D_1F_2 - D_2D_2F_1 - D_3D_3F_1 + D_3D_1F_3 \\
      D_3D_2F_3 - D_3D_3F_2 - D_1D_1F_2 + D_1D_2F_1 \\
      D_1D_3F_1 - D_1D_1F_3 - D_2D_2F_3 + D_2D_3F_2
    } \\ &=
    (D_1^2 + D_2^2 + D_3^2)\bmat{F_1 \\ F_2 \\ F_3} \\ &=
    \bmat{\Delta F_1 \\ \Delta F_2 \\ \Delta F_3}
  \end{align*}
\end{problem}

\end{document}
